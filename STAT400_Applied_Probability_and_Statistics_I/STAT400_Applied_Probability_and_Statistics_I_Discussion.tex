\documentclass{article}
\usepackage{graphicx} % Required for inserting images
\usepackage{amsmath}
\usepackage{amssymb}   % Additional symbols, such as \mathbb and various math symbols
\usepackage{amsfonts}  % Additional fonts for mathematical symbols, like \mathbb
\usepackage{mathtools} % Extends amsmath with additional functionality
\usepackage{mathrsfs}  % Provides script fonts (\mathscr)
\usepackage{amsthm}    % Enhanced theorem environments (theorem, lemma, etc.)
\usepackage{bbm}
\usepackage{tikz}
\usetikzlibrary{shapes.geometric, calc}
\usepackage{array}
\usetikzlibrary{intersections, calc}
\usepackage{geometry}

\title{Applied Probability and Statistics I Discussion - STAT400}
\author{Tom Mitchell}
\date{Yi - Fall 2024}

\begin{document}

\maketitle

\section*{Syllabus}

\subsection*{Grading}
\begin{itemize}
    \item Homework — 28\% (7\% each)
    \item R Projects — 12\% (4\% each)
    \item Two exams — 30\% (15\% each)
    \item Final exam — 30\%
\end{itemize}

\subsection*{Office Hours}
\begin{itemize}
    \item Tuesday: 1:00 PM - 1:50 PM (in person, MTH 4106)
    \item Wednesday: 11:00 AM - 11:50 AM (online)
\end{itemize}

\subsection*{Exams}
\begin{itemize}
    \item 2 midterms and a final exam
\end{itemize}

\section*{Discussion 1: Monday 9/5/2024}

\subsection*{Notation and Key Concepts}

\begin{itemize}
    \item Cardinality: For a finite set A, the number of elements is denoted as $|A|$.
    \item Example: If A = \{1, 2, 3, 4, 5\}, then $|A| = 5$.
\end{itemize}

\subsection*{Probability Axioms}

The total probability of all outcomes in a sample space must sum to 1:

\begin{enumerate}
    \item For discrete probability: $\sum_{i} P_i = 1$
    \item For continuous probability: $\int p(x) \, dx = 1$
\end{enumerate}

\subsection*{Homework 1}




1. Biased Coin Experiment

A biased coin comes up heads with probability $\frac{1}{3}$. An experiment consists of tossing this coin until heads is seen for the first time, at which point the experiment ends. The probability of seeing heads on the $i$th toss is $\frac{1}{3} \left(\frac{2}{3}\right)^{i-1}$.

Probability table:
\begin{center}
\begin{tabular}{|c|c|c|}
\hline
Outcome & H & T \\
\hline
Probability & $\frac{1}{3}$ & $\frac{2}{3}$ \\
\hline
\end{tabular}
\end{center}

Compute the probability that you will see heads for the following scenarios, clearly identifying the pairwise disjoint events used in the calculations:

a) H (on first toss)
b) TH (on second toss)
c) TTTH (on fourth toss)
d) TTT...TH (on $i$th toss)

Note: The $i$th toss ending the game corresponds to seeing heads on the $i$th toss.

Derive the formula for the probability of seeing heads on the $i$th toss:

$P(\text{H on }i\text{th toss}) = \frac{1}{3} \left(\frac{2}{3}\right)^{i-1}$

This formula represents the probability of getting tails $(i-1)$ times followed by heads on the $i$th toss. This is already given in the problem statement.

\subsection*{(a) Probability of heads on the 1st, 3rd, or 7th toss}
Let $H_i$ denote the event that heads occurs on the $i$th toss. The events $H_1$, $H_3$, and $H_7$ are pairwise disjoint.

\[
P(\text{heads on 1st, 3rd, or 7th toss}) = P(H_1 \cup H_3 \cup H_7)
\]
Since the events are pairwise disjoint:
\[
P(H_1 \cup H_3 \cup H_7) = P(H_1) + P(H_3) + P(H_7)
\]
Using the given probabilities:
\[
P(H_1) = \frac{1}{3}, \quad P(H_3) = \frac{1}{3} \left(\frac{2}{3}\right)^2, \quad P(H_7) = \frac{1}{3} \left(\frac{2}{3}\right)^6
\]
Thus:
\[
P(H_1 \cup H_3 \cup H_7) = \frac{1}{3} + \frac{1}{3} \cdot \frac{4}{9} + \frac{1}{3} \cdot \frac{64}{729} = \boxed{\frac{1117}{2187}}
\]

\subsection*{(b) Probability of heads on an odd-numbered toss}
Let $H_{\text{odd}}$ denote the event that heads occurs on an odd-numbered toss. The set of odd-numbered tosses is the union of pairwise disjoint events:
\[
P(H_{\text{odd}}) = P(H_1 \cup H_3 \cup H_5 \cup \dots)
\]
Using the geometric series:
\[
P(H_{\text{odd}}) = \sum_{k=0}^{\infty} \frac{1}{3} \left(\frac{2}{3}\right)^{2k}
\]
This sum can be simplified as:
\[
P(H_{\text{odd}}) = \frac{\frac{1}{3}}{1 - \left(\frac{2}{3}\right)^2} = \frac{\frac{1}{3}}{1 - \frac{4}{9}} = \frac{1}{3} \cdot \frac{9}{5} = \boxed{\frac{3}{5}}
\]

\section*{Problem 5}
Suppose that $\Omega = \{\omega_1, \ldots, \omega_n\}$ be a finite sample space with $n > 2$ elements. Suppose $P(\{\omega_1\}) = \frac{1}{2}$ and $P(\{\omega_i\}) = P(\{\omega_j\})$ for all $i, j \neq 1$.


\begin{table}[h]
\centering
\begin{tabular}{|c|c|c|c|}
\hline
$\omega_1$ & $\omega_2$ & $\cdots$ & $\omega_n$ \\
\hline
$\frac{1}{2}$ & $x$ & $\cdots$ & $x$ \\
\hline
\end{tabular}
\caption{Probability distribution of sample space $\Omega$}
\end{table}

\noindent Where:
\begin{enumerate}
    \item $P(\{\omega_1\}) = \frac{1}{2}$
    \item $P(\{\omega_i\}) = P(\{\omega_j\})$ for all $i, j \neq 1$
\end{enumerate}

\noindent \textbf{Note on disjointness:}
\begin{itemize}
    \item Pairwise disjoint: $A_i \cap A_j = \emptyset$ for all $i \neq j$
    \item Disjoint: $\bigcap_{i=1}^n A_i = \emptyset$ for all $i, j$ such that $i \neq j$
\end{itemize}



\subsection*{(a) Show that $P(\{\omega_i\}) = \frac{1}{2(n-1)}$ for $i = 2, \ldots, n$}
Let $P(\{\omega_i\}) = x$ for $i = 2, \ldots, n$. We know:
\[
P(\Omega) = 1 \quad \text{and} \quad P(\{\omega_1\}) + (n-1)x = 1
\]
Substituting $P(\{\omega_1\}) = \frac{1}{2}$:
\[
\frac{1}{2} + (n-1)x = 1 \quad \Rightarrow \quad (n-1)x = \frac{1}{2} \quad \Rightarrow \quad x = \frac{1}{2(n-1)}
\]

\subsection*{(b) Construct a formula for computing $P(A)$ for any $A \subseteq \Omega$}


We can consider two cases ($x$ is defined as found in part a):

\textbf{Case 1}: $\omega_1 \in A$

If $\omega_1 \in A$, then:

\[
P(A) = \frac{1}{2} + |A \cap \{\omega_2, \ldots, \omega_n\}| \cdot x = \frac{1}{2} + \frac{|A \cap \{\omega_2, \ldots, \omega_n\}|}{2(n-1)}.
\]

\textbf{Case 2}: $\omega_1 \notin A$

If $\omega_1 \notin A$, then:

\[
P(A) = |A \cap \{\omega_2, \ldots, \omega_n\}| \cdot x = \frac{|A|}{2(n-1)} = \frac{|A \cap \{\omega_2, \ldots, \omega_n\}|}{2(n-1)}.
\]

\subsection*{Final Formula}

Thus, the formula for $P(A)$ is:

\[
P(A) = 
\begin{cases} 
\frac{1}{2} + \frac{|A \cap \{\omega_2, \ldots, \omega_n\}|}{2(n-1)} & \text{if } \omega_1 \in A, \\[10pt]
\frac{|A \cap \{\omega_2, \ldots, \omega_n\}|}{2(n-1)} & \text{if } \omega_1 \notin A.
\end{cases}
\]


















\end{document}